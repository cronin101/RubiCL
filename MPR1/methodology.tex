\section{Methodology}
The framework uses a Ruby library with a C native extension that interfaces with the OpenCL API for low level calculations.
Presenting a high level abstraction is handled by the Ruby code as its dynamic nature permitting metaprogramming facilitates the construction of abstractions useful for creating Domain Specific APIs.

It is felt that C is the wisest choice for this project's low-level implementation. This is due to the lack of some abstraction features benefiting code-clarity. 

It is easy to produce descriptive assertion tests in the \emph{Ruby} language. Good test-coverage of the library's computations will provide greater confidence in the correctness of the code's output each time that changes are made. With a set of tests specifying correct outputs over sample computations; it is much easier to change the architecture when attempting optimisation, as you can assert that the resulting values are unchanged.

Providing access to high-performance computation on data-sets via low-level extensions is advantageous to such languages as they often suffer compared to less syntactically-expressive languages when operating on basic types. The library should allow efficient computation without the programmer having to worry about implementing efficient data-types themselves.

The testing of OpenCL interaction will be performed on a variety of devices throughout it's development, ranging from a laptop's \emph{HD4000} integrated GPU to a high-end GPU in a current generation desktop machine. This makes it possible to investigate how optimisations affect differing architectures and hints at which benefits can be shared throughout all devices.
