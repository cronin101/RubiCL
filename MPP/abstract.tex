\begin{abstract}       % subsection 1.1.
  This proposal outlines a project researching the implementation of large-scale data-processing paradigms on multi-core systems.
  The aim is to provide easily-exploited access to the vast number of programmable cores present in modern systems containing \ac{GPUs} and investigate how their usage can be optimised.

  The project should provide a framework that will be of use to anyone who has access to a high-performance GPU device and would benefit from an increase in throughput when processing datasets in a manner that can be suitably parallelised.
  In addition, the project hopes to avoid the disadvantages of similar parallel frameworks that require large amounts of user-intervention to tune execution. This need for task-specific configuration increase the barriers to entry of otherwise obtainable benefits.

These goals concentrate on maximising the performance and usability of a framework that results from evaluation and iteration of optimisations alongside research into existing failings.

The resulting implementation shall aim to verify the hypothesis that there exists use-cases that can be efficiently implemented on a highly parallel device whilst remaining programmer-friendly.
This abstraction may be achieved by presenting a collection of simple functions that encapsulate all required complexities.

Researching relevant advances, producing the framework, and evaluating it against similar project outcomes should require a lengthy period of management and iteration.
This document should demonstrate that there is sufficient depth in research plans, suitable for a long period of focussed activity.
\end{abstract}


